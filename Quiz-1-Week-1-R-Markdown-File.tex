\PassOptionsToPackage{unicode=true}{hyperref} % options for packages loaded elsewhere
\PassOptionsToPackage{hyphens}{url}
%
\documentclass[]{article}
\usepackage{lmodern}
\usepackage{amssymb,amsmath}
\usepackage{ifxetex,ifluatex}
\usepackage{fixltx2e} % provides \textsubscript
\ifnum 0\ifxetex 1\fi\ifluatex 1\fi=0 % if pdftex
  \usepackage[T1]{fontenc}
  \usepackage[utf8]{inputenc}
  \usepackage{textcomp} % provides euro and other symbols
\else % if luatex or xelatex
  \usepackage{unicode-math}
  \defaultfontfeatures{Ligatures=TeX,Scale=MatchLowercase}
\fi
% use upquote if available, for straight quotes in verbatim environments
\IfFileExists{upquote.sty}{\usepackage{upquote}}{}
% use microtype if available
\IfFileExists{microtype.sty}{%
\usepackage[]{microtype}
\UseMicrotypeSet[protrusion]{basicmath} % disable protrusion for tt fonts
}{}
\IfFileExists{parskip.sty}{%
\usepackage{parskip}
}{% else
\setlength{\parindent}{0pt}
\setlength{\parskip}{6pt plus 2pt minus 1pt}
}
\usepackage{hyperref}
\hypersetup{
            pdftitle={Quiz 1 Week 1 R Markdown file},
            pdfauthor={Me},
            pdfborder={0 0 0},
            breaklinks=true}
\urlstyle{same}  % don't use monospace font for urls
\usepackage[margin=1in]{geometry}
\usepackage{color}
\usepackage{fancyvrb}
\newcommand{\VerbBar}{|}
\newcommand{\VERB}{\Verb[commandchars=\\\{\}]}
\DefineVerbatimEnvironment{Highlighting}{Verbatim}{commandchars=\\\{\}}
% Add ',fontsize=\small' for more characters per line
\usepackage{framed}
\definecolor{shadecolor}{RGB}{248,248,248}
\newenvironment{Shaded}{\begin{snugshade}}{\end{snugshade}}
\newcommand{\AlertTok}[1]{\textcolor[rgb]{0.94,0.16,0.16}{#1}}
\newcommand{\AnnotationTok}[1]{\textcolor[rgb]{0.56,0.35,0.01}{\textbf{\textit{#1}}}}
\newcommand{\AttributeTok}[1]{\textcolor[rgb]{0.77,0.63,0.00}{#1}}
\newcommand{\BaseNTok}[1]{\textcolor[rgb]{0.00,0.00,0.81}{#1}}
\newcommand{\BuiltInTok}[1]{#1}
\newcommand{\CharTok}[1]{\textcolor[rgb]{0.31,0.60,0.02}{#1}}
\newcommand{\CommentTok}[1]{\textcolor[rgb]{0.56,0.35,0.01}{\textit{#1}}}
\newcommand{\CommentVarTok}[1]{\textcolor[rgb]{0.56,0.35,0.01}{\textbf{\textit{#1}}}}
\newcommand{\ConstantTok}[1]{\textcolor[rgb]{0.00,0.00,0.00}{#1}}
\newcommand{\ControlFlowTok}[1]{\textcolor[rgb]{0.13,0.29,0.53}{\textbf{#1}}}
\newcommand{\DataTypeTok}[1]{\textcolor[rgb]{0.13,0.29,0.53}{#1}}
\newcommand{\DecValTok}[1]{\textcolor[rgb]{0.00,0.00,0.81}{#1}}
\newcommand{\DocumentationTok}[1]{\textcolor[rgb]{0.56,0.35,0.01}{\textbf{\textit{#1}}}}
\newcommand{\ErrorTok}[1]{\textcolor[rgb]{0.64,0.00,0.00}{\textbf{#1}}}
\newcommand{\ExtensionTok}[1]{#1}
\newcommand{\FloatTok}[1]{\textcolor[rgb]{0.00,0.00,0.81}{#1}}
\newcommand{\FunctionTok}[1]{\textcolor[rgb]{0.00,0.00,0.00}{#1}}
\newcommand{\ImportTok}[1]{#1}
\newcommand{\InformationTok}[1]{\textcolor[rgb]{0.56,0.35,0.01}{\textbf{\textit{#1}}}}
\newcommand{\KeywordTok}[1]{\textcolor[rgb]{0.13,0.29,0.53}{\textbf{#1}}}
\newcommand{\NormalTok}[1]{#1}
\newcommand{\OperatorTok}[1]{\textcolor[rgb]{0.81,0.36,0.00}{\textbf{#1}}}
\newcommand{\OtherTok}[1]{\textcolor[rgb]{0.56,0.35,0.01}{#1}}
\newcommand{\PreprocessorTok}[1]{\textcolor[rgb]{0.56,0.35,0.01}{\textit{#1}}}
\newcommand{\RegionMarkerTok}[1]{#1}
\newcommand{\SpecialCharTok}[1]{\textcolor[rgb]{0.00,0.00,0.00}{#1}}
\newcommand{\SpecialStringTok}[1]{\textcolor[rgb]{0.31,0.60,0.02}{#1}}
\newcommand{\StringTok}[1]{\textcolor[rgb]{0.31,0.60,0.02}{#1}}
\newcommand{\VariableTok}[1]{\textcolor[rgb]{0.00,0.00,0.00}{#1}}
\newcommand{\VerbatimStringTok}[1]{\textcolor[rgb]{0.31,0.60,0.02}{#1}}
\newcommand{\WarningTok}[1]{\textcolor[rgb]{0.56,0.35,0.01}{\textbf{\textit{#1}}}}
\usepackage{graphicx,grffile}
\makeatletter
\def\maxwidth{\ifdim\Gin@nat@width>\linewidth\linewidth\else\Gin@nat@width\fi}
\def\maxheight{\ifdim\Gin@nat@height>\textheight\textheight\else\Gin@nat@height\fi}
\makeatother
% Scale images if necessary, so that they will not overflow the page
% margins by default, and it is still possible to overwrite the defaults
% using explicit options in \includegraphics[width, height, ...]{}
\setkeys{Gin}{width=\maxwidth,height=\maxheight,keepaspectratio}
\setlength{\emergencystretch}{3em}  % prevent overfull lines
\providecommand{\tightlist}{%
  \setlength{\itemsep}{0pt}\setlength{\parskip}{0pt}}
\setcounter{secnumdepth}{0}
% Redefines (sub)paragraphs to behave more like sections
\ifx\paragraph\undefined\else
\let\oldparagraph\paragraph
\renewcommand{\paragraph}[1]{\oldparagraph{#1}\mbox{}}
\fi
\ifx\subparagraph\undefined\else
\let\oldsubparagraph\subparagraph
\renewcommand{\subparagraph}[1]{\oldsubparagraph{#1}\mbox{}}
\fi

% set default figure placement to htbp
\makeatletter
\def\fps@figure{htbp}
\makeatother


\title{Quiz 1 Week 1 R Markdown file}
\author{Me}
\date{4/5/2020}

\begin{document}
\maketitle

\hypertarget{reading-in-the-data}{%
\subsection{Reading in the data}\label{reading-in-the-data}}

\hypertarget{reading-in-the-data-method-1}{%
\subsubsection{Reading in the data method
1}\label{reading-in-the-data-method-1}}

\begin{Shaded}
\begin{Highlighting}[]
\KeywordTok{getwd}\NormalTok{()}
\end{Highlighting}
\end{Shaded}

\begin{verbatim}
## [1] "/Users/sadikafhameed/Documents/Coursera/R Programming /Week 1/Quiz 1"
\end{verbatim}

\begin{Shaded}
\begin{Highlighting}[]
\NormalTok{data <-}\StringTok{ }\KeywordTok{read.csv}\NormalTok{(}\StringTok{"hw1_data.csv"}\NormalTok{)}
\end{Highlighting}
\end{Shaded}

\hypertarget{reading-in-the-data-method-2}{%
\subsubsection{Reading in the data method
2}\label{reading-in-the-data-method-2}}

\begin{Shaded}
\begin{Highlighting}[]
\NormalTok{dat <-}\StringTok{ }\KeywordTok{download.file}\NormalTok{(}\StringTok{'https://d396qusza40orc.cloudfront.net/rprog/data/quiz1_data.zip'}\NormalTok{, }\DataTypeTok{destfile =}\StringTok{"quizdat.zip"}\NormalTok{)}
\NormalTok{dat <-}\StringTok{ }\KeywordTok{unzip}\NormalTok{(}\StringTok{"quizdat.zip"}\NormalTok{)}
\NormalTok{dat <-}\StringTok{ }\KeywordTok{read.csv}\NormalTok{(}\StringTok{"hw1_data.csv"}\NormalTok{)}
\end{Highlighting}
\end{Shaded}

\hypertarget{in-the-dataset-provided-for-this-quiz-what-are-the-column-names-of-the-dataset}{%
\subsection{In the dataset provided for this Quiz, what are the column
names of the
dataset?}\label{in-the-dataset-provided-for-this-quiz-what-are-the-column-names-of-the-dataset}}

\hypertarget{meethod-1}{%
\subsubsection{MEethod 1}\label{meethod-1}}

\begin{Shaded}
\begin{Highlighting}[]
\KeywordTok{colnames}\NormalTok{(data)}
\end{Highlighting}
\end{Shaded}

\begin{verbatim}
## [1] "Ozone"   "Solar.R" "Wind"    "Temp"    "Month"   "Day"
\end{verbatim}

\hypertarget{method-2}{%
\subsubsection{Method 2}\label{method-2}}

\begin{Shaded}
\begin{Highlighting}[]
\KeywordTok{head}\NormalTok{(data)}
\end{Highlighting}
\end{Shaded}

\begin{verbatim}
##   Ozone Solar.R Wind Temp Month Day
## 1    41     190  7.4   67     5   1
## 2    36     118  8.0   72     5   2
## 3    12     149 12.6   74     5   3
## 4    18     313 11.5   62     5   4
## 5    NA      NA 14.3   56     5   5
## 6    28      NA 14.9   66     5   6
\end{verbatim}

\hypertarget{method-3}{%
\subsubsection{Method 3}\label{method-3}}

\begin{Shaded}
\begin{Highlighting}[]
\KeywordTok{names}\NormalTok{(dat)}
\end{Highlighting}
\end{Shaded}

\begin{verbatim}
## [1] "Ozone"   "Solar.R" "Wind"    "Temp"    "Month"   "Day"
\end{verbatim}

\hypertarget{extract-the-first-2-rows-of-the-data-frame-and-print-them-to-the-console.-what-does-the-output-look-like}{%
\subsection{Extract the first 2 rows of the data frame and print them to
the console. What does the output look
like?}\label{extract-the-first-2-rows-of-the-data-frame-and-print-them-to-the-console.-what-does-the-output-look-like}}

\hypertarget{method-1}{%
\subsubsection{Method 1}\label{method-1}}

\begin{Shaded}
\begin{Highlighting}[]
\NormalTok{dat[}\DecValTok{1}\OperatorTok{:}\DecValTok{2}\NormalTok{, ]}
\end{Highlighting}
\end{Shaded}

\begin{verbatim}
##   Ozone Solar.R Wind Temp Month Day
## 1    41     190  7.4   67     5   1
## 2    36     118  8.0   72     5   2
\end{verbatim}

\hypertarget{method-2-1}{%
\subsubsection{Method 2}\label{method-2-1}}

\begin{Shaded}
\begin{Highlighting}[]
\NormalTok{dat[}\DecValTok{152}\OperatorTok{:}\DecValTok{153}\NormalTok{, ]}
\end{Highlighting}
\end{Shaded}

\begin{verbatim}
##     Ozone Solar.R Wind Temp Month Day
## 152    18     131  8.0   76     9  29
## 153    20     223 11.5   68     9  30
\end{verbatim}

\hypertarget{method-3-1}{%
\subsubsection{Method 3}\label{method-3-1}}

\begin{Shaded}
\begin{Highlighting}[]
\KeywordTok{tail}\NormalTok{(data, }\DecValTok{2}\NormalTok{)}
\end{Highlighting}
\end{Shaded}

\begin{verbatim}
##     Ozone Solar.R Wind Temp Month Day
## 152    18     131  8.0   76     9  29
## 153    20     223 11.5   68     9  30
\end{verbatim}

\hypertarget{how-many-observations-i.e.rows-are-in-this-data-frame}{%
\subsection{How many observations (i.e.~rows) are in this data
frame?}\label{how-many-observations-i.e.rows-are-in-this-data-frame}}

\begin{Shaded}
\begin{Highlighting}[]
\KeywordTok{nrow}\NormalTok{(data)}
\end{Highlighting}
\end{Shaded}

\begin{verbatim}
## [1] 153
\end{verbatim}

\hypertarget{question-15-what-is-the-value-of-ozone-in-the-47th-row}{%
\subsection{Question 15 What is the value of Ozone in the 47th
row?}\label{question-15-what-is-the-value-of-ozone-in-the-47th-row}}

\hypertarget{method-1-1}{%
\subsubsection{Method 1}\label{method-1-1}}

``` data{[}47, {]}

\hypertarget{section}{%
\subsubsection{}\label{section}}

\begin{Shaded}
\begin{Highlighting}[]
\NormalTok{dat}\OperatorTok{$}\NormalTok{Ozone[}\DecValTok{47}\NormalTok{]}
\end{Highlighting}
\end{Shaded}

\begin{verbatim}
## [1] 21
\end{verbatim}

\hypertarget{how-many-missing-values-are-in-the-ozone-column-of-this-data-frame}{%
\subsection{How many missing values are in the Ozone column of this data
frame?}\label{how-many-missing-values-are-in-the-ozone-column-of-this-data-frame}}

\hypertarget{method-1-2}{%
\subsubsection{Method 1}\label{method-1-2}}

\begin{Shaded}
\begin{Highlighting}[]
\KeywordTok{sum}\NormalTok{(}\KeywordTok{is.na}\NormalTok{(dat}\OperatorTok{$}\NormalTok{Ozone))}
\end{Highlighting}
\end{Shaded}

\begin{verbatim}
## [1] 37
\end{verbatim}

\hypertarget{method-2-2}{%
\subsubsection{Method 2}\label{method-2-2}}

\begin{Shaded}
\begin{Highlighting}[]
\NormalTok{sub <-}\StringTok{ }\KeywordTok{subset}\NormalTok{(data, }\KeywordTok{is.na}\NormalTok{(Ozone))}
\KeywordTok{nrow}\NormalTok{(sub)}
\end{Highlighting}
\end{Shaded}

\begin{verbatim}
## [1] 37
\end{verbatim}

\hypertarget{method-3-2}{%
\subsubsection{Method 3}\label{method-3-2}}

\hypertarget{to-calculate-as-an-example-how-many-missing-values-there-are-in-total}{%
\paragraph{To calculate as an example how many missing values there are
in
total}\label{to-calculate-as-an-example-how-many-missing-values-there-are-in-total}}

\begin{Shaded}
\begin{Highlighting}[]
\KeywordTok{length}\NormalTok{(}\KeywordTok{which}\NormalTok{(}\KeywordTok{is.na}\NormalTok{((data))))}
\end{Highlighting}
\end{Shaded}

\begin{verbatim}
## [1] 44
\end{verbatim}

\hypertarget{to-calculate-only-for-ozone}{%
\paragraph{To calculate only for
Ozone}\label{to-calculate-only-for-ozone}}

\begin{verbatim}
datana <- subset(dat, is.na(Ozone))
nrow(datana)

## What is the mean of the Ozone column in this dataset? Exclude missing values (coded as NA) from this calculation.

### Method 1
\end{verbatim}

datanotna \textless{}- subset(dat, !is.na(Ozone)) apply(datanotna, 2,
mean)

\hypertarget{method-2-3}{%
\subsubsection{Method 2}\label{method-2-3}}

``` mean(dat\$Ozone, na.rm = TRUE) \#\#\# Method 3

\begin{Shaded}
\begin{Highlighting}[]
\NormalTok{sub =}\StringTok{ }\KeywordTok{subset}\NormalTok{(data, }\OperatorTok{!}\KeywordTok{is.na}\NormalTok{(Ozone), }\DataTypeTok{select =}\NormalTok{ Ozone)}
\KeywordTok{apply}\NormalTok{(sub, }\DecValTok{2}\NormalTok{, mean)}
\end{Highlighting}
\end{Shaded}

\begin{verbatim}
##    Ozone 
## 42.12931
\end{verbatim}

\hypertarget{extract-the-subset-of-rows-of-the-data-frame-where-ozone-values-are-above-31-and-temp-values-are-above-90.-what-is-the-mean-of-solar.r-in-this-subset}{%
\subsection{Extract the subset of rows of the data frame where Ozone
values are above 31 and Temp values are above 90. What is the mean of
Solar.R in this
subset?}\label{extract-the-subset-of-rows-of-the-data-frame-where-ozone-values-are-above-31-and-temp-values-are-above-90.-what-is-the-mean-of-solar.r-in-this-subset}}

\hypertarget{method-1-3}{%
\subsubsection{Method 1}\label{method-1-3}}

\begin{Shaded}
\begin{Highlighting}[]
\NormalTok{sub <-}\StringTok{ }\KeywordTok{subset}\NormalTok{(data, Ozone }\OperatorTok{>}\StringTok{ }\DecValTok{31} \OperatorTok{&}\StringTok{ }\NormalTok{Temp }\OperatorTok{>}\DecValTok{90}\NormalTok{, }\DataTypeTok{select =}\NormalTok{ Solar.R)}
\KeywordTok{apply}\NormalTok{(sub, }\DecValTok{2}\NormalTok{, mean)}
\end{Highlighting}
\end{Shaded}

\begin{verbatim}
## Solar.R 
##   212.8
\end{verbatim}

\hypertarget{method-2-4}{%
\subsubsection{Method 2}\label{method-2-4}}

\begin{Shaded}
\begin{Highlighting}[]
\KeywordTok{mean}\NormalTok{(dat[}\KeywordTok{which}\NormalTok{(dat}\OperatorTok{$}\NormalTok{Ozone }\OperatorTok{>}\DecValTok{31} \OperatorTok{&}\StringTok{ }\NormalTok{dat}\OperatorTok{$}\NormalTok{Temp }\OperatorTok{>}\StringTok{ }\DecValTok{90}\NormalTok{),]}\OperatorTok{$}\NormalTok{Solar.R)}
\end{Highlighting}
\end{Shaded}

\begin{verbatim}
## [1] 212.8
\end{verbatim}

\hypertarget{method-3-3}{%
\subsubsection{Method 3}\label{method-3-3}}

\begin{Shaded}
\begin{Highlighting}[]
\NormalTok{datasub <-}\StringTok{ }\KeywordTok{subset}\NormalTok{(dat, dat}\OperatorTok{$}\NormalTok{Ozone }\OperatorTok{>}\StringTok{ }\DecValTok{31} \OperatorTok{&}\StringTok{ }\NormalTok{dat}\OperatorTok{$}\NormalTok{Temp }\OperatorTok{>}\DecValTok{90}\NormalTok{, }\DataTypeTok{select =}\NormalTok{ Solar.R)}
\KeywordTok{apply}\NormalTok{(datasub, }\DecValTok{2}\NormalTok{, mean)}
\end{Highlighting}
\end{Shaded}

\begin{verbatim}
## Solar.R 
##   212.8
\end{verbatim}

\hypertarget{what-is-the-mean-of-temp-when-month-is-equal-to-6}{%
\subsection{What is the mean of ``Temp'' when ``Month'' is equal to
6?}\label{what-is-the-mean-of-temp-when-month-is-equal-to-6}}

\hypertarget{method-1-4}{%
\subsubsection{Method 1}\label{method-1-4}}

\begin{verbatim}
mean(dat[which(dat$Month == 6), ]$Temp)
\end{verbatim}

\hypertarget{method-2-5}{%
\subsubsection{Method 2}\label{method-2-5}}

\begin{verbatim}
sub = subset(dat, Month == 6, select = Temp)
apply(sub, 2, mean)
\end{verbatim}

\hypertarget{method-3-4}{%
\subsubsection{Method 3}\label{method-3-4}}

\begin{verbatim}
datasub <- subset(data, data$Month == 6, select = Temp)
apply(datasub, 2 , mean)
\end{verbatim}

\hypertarget{what-was-the-maximum-ozone-value-in-the-month-of-may-i.e.month-5}{%
\subsection{What was the maximum ozone value in the month of May
(i.e.~Month =
5)}\label{what-was-the-maximum-ozone-value-in-the-month-of-may-i.e.month-5}}

\hypertarget{method-1-5}{%
\subsubsection{Method 1}\label{method-1-5}}

\begin{verbatim}
datasub <- subset(data, !is.na(Ozone) & data$Month ==5, select = Ozone)
apply(datasub, 2, max)
\end{verbatim}

\hypertarget{method-2-6}{%
\subsubsection{Method 2}\label{method-2-6}}

\begin{verbatim}
sub <- subset(data, Month == 5 & !is.na(Ozone), select = Ozone       )
apply(sub, 2, max)
\end{verbatim}

\end{document}
